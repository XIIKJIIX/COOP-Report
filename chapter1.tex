\chapter{บทนำ}
\label{chapter:introduction}

\section{ที่มาและความสำคัญ}

ในช่วงสองถึงสามปีที่ผ่านมา ข่าวด้านความปลอดภัยทางคอมพิวเตอร์ได้เกิดขึ้นบ่อยครั้ง ไม่ว่าจะเป็นข่าวที่เกิดขึ้นกับบุคคลธรรมดาสามัญ ที่มีผู้ที่ไม่หวังดีขโมยรหัสผ่าน Facebook ผ่านช่องทางหลอกลวง เช่น หลอกให้ผู้ใช้งาน Facebook ตอบกระทู้ด้วยรหัสผ่านบัญชีผู้ใช้ของตัวเอง เพื่อสุ่มแจกของรางวัล ทั้งที่จริง ๆ แล้วเป็นการหลอกลวง และได้นำรหัสผ่านนั้นไปใช้ปลอมตัวเป็นเจ้าของบัญชี เพื่อที่จะไปหลอกยืมเงินผู้อื่น ซึ่งทำให้เจ้าของบัญชีที่แท้จริงนั้นสูญเสียความน่าเชื่อถือ และเกิดความเข้าใจผิดขึ้น หรือแม้แต่บริษัทการสื่อสารแห่งหนึ่งที่ดูน่าเชื่อถือก็สามารถละเลยเรื่องความปลอดภัยของระบบจนทำให้ข้อมูลรูปสำเนาบัตรประชาชนเปิดเผยบนอินเตอร์เน็ตโดยที่ไม่มีการป้องกันใด ๆ ซึ่งอาจจะมีผู้ที่ไม่หวังดีนำข้อมูลไปใช้ปลอมแปลงเอกสารได้ ท้ายที่สุดมีผู้ประสงค์ดีได้รายงานไปยังบริษัท แต่ก็ต้องใช้เวลาถึงหนึ่งเดือนในการแก้ไขตั้งค่าข้อมูลไม่ให้สาธารณะเข้าถึงได้ \cite{truemoveh}

จะเห็นว่าภัยอันตรายทางคอมพิวเตอร์สามารถประสบพบเจอกับตัว ได้ทั้งทางตรงและทางอ้อม กล่าวคือถ้าเราไม่หลงกลเปิดแผยรหัสผ่านส่วนตัวให้กับผู้อื่น ก็อาจจะมีบริษัทที่เราเป็นลูกค้าทำข้อมูลส่วนตัวเราหลุดรอดออกมาสู่โลกอินเตอร์เน็ตด้วยความผิดพลาด ดังนั้นการที่เรามีความรู้ติดตัวด้านความปลอดภัยของคอมพิวเตอร์ไว้ก็จะช่วยลดความเสี่ยงที่จะเกิดผลกระทบทางตรงได้ เช่น ตั้งรหัสผ่านแต่ละบัญชีไม่ซ้ำกันและไม่ใช้วันเกิดหรือข้อมูลส่วนตัวเป็นส่วนประกอบของรหัสผ่าน เพื่อให้ยากต่อการเดาของผู้ไม่หวังดี  รู้จักวิธีแยกแยะจดหมายอิเล็กทรอนิกส์ที่เป็นประเภทหลอกลวง หรือแอบแฝงโปรแกรมที่ไม่หวังดี ซึ่งองค์ความรู้เหล่านี้สามารถหาได้ตามอินเทอร์เน็ต ส่วนวิธีการลดความเสี่ยงที่จะเกิดผลกระทบทางอ้อมกับตัวเรานั้นองค์กรหรือหน่วยงานรัฐและเอกชนจะต้องหมั่นตรวจสอบและดูแลรักษาความปลอดภัยระบบสารสนเทศของตัวเองไม่ให้มีช่องโหว่ และหมั่นค้นคว้าข่าวสารด้านความปลอดภัยและนำมาปรับปรุงให้กับระบบของตัวเองอยู่เสมอ 

ซึ่งงานของผู้เขียนนั้นเป็นงานค้นหาช่องโหว่ในระบบสารสนเทศของลูกค้า ทั้งใช้งานภายในและสาธารณะที่มีผู้ใช้งานจำนวนมาก

ในรายงานนี้มีจุดมุ่งหมายเพื่อรวบรวมช่องโหว่บน Web Platform ที่พบจอบ่อย ๆ จากการปฏิบัติงานและนำมาจัดทำรายงาน โดยจัดเป็นหมวดหมู่ตาม OWASP Top 10 เพื่อให้ผู้ดูแลหรือผู้พัฒนาระบบเทคโนโลยีสารสนเทศในหน่วยงานหรือองค์กรต่าง ๆ และผู้เริ่มต้นศึกษาด้านความปลอดภัยของคอมพิวเตอร์ศึกษาข้อมูลเป็นกรณีศึกษา และนำไปปรับใช้ ปรับปรุง และแก้ไขระบบในหน่วยงานตนเอง

\newpage
\section{วัตถุประสงค์ของการปฏิบัติงาน}
\begin{enumerate}
	\item เพื่อให้ผู้ดูแลหรือพัฒนาระบบในหน่วยงานหรือองค์กรต่าง ๆ ได้นำวิธีการตรวจสอบช่องโหว่จากรายงานฉบับนี้ไปปรับปรุงแก้ไขระบบของตน
	\item เพื่อเพิ่มพนูประสบการณ์ ความรู้ เทคนิค และการแก้ปัญหาจากการปฏิบัติงานจริง
\end{enumerate}

\section{ขอบเขตของการปฏิบัติงาน}
\begin{enumerate}
    \item ทดสอบเจาะระบบ และแนะนำวิธีการในการแก้ไขและป้องกันช่องโหว่ ตามขอบเขตของระบบที่ลูกค้าต้องการ
\end{enumerate}

